\documentclass[oneside]{book}
\usepackage[utf8]{inputenc}
\usepackage{float}
\usepackage{graphicx}
\usepackage{amsmath}
\usepackage{color}
\usepackage{multicol}
\usepackage{ragged2e}
\usepackage{listings}
\usepackage{pdfpages}
\title{Notes de Cours INF 8225}
\date{2018-01-01}
\author{Olivier Sirois}
\setlength\parindent{0pt}
\makeindex
\pagenumbering{arabic}
\begin{document}
\setcounter{page}{1}
\maketitle
\tableofcontents

\section{Premier cours  + plan de cours}
Premier cours réutilise beaucoup les concepts des premiers chapitre d'INF 8215. i.e :  les agents intelligents avec leur définition (voir chap. 1-2-3 du INF8215).  
\chapter{1 - Méthodes probabiliste, Maximum Likelihood, Apprentissage automatique}
Practical machine learning - chapter 9, Appendix A, 

\chapter{2 - Réseaux bayésien, méthode probabiliste, autres.}
\section{Intro au cours}
\end{document}